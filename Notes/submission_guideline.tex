%!TEX program=lualatex
%!TEX options=-synctex=1 -interaction=nonstopmode -halt-on-error "%DOC%.tex"
\documentclass[11pt,pdfa,lastpage,minititle]{MishoNote}
\title{Tips for Writing and Submitting Assignments}
\author{Sho Iwamoto}
\hypersetup{
  pdflang={en-US},
  pdfauthortitle={Assistant Professor, National Sun Yat-sen University},
  pdfsubject={Sho's Suggestions on how to handle assignments and reports for university students.},
  pdfcontactemail={iwamoto@g-mail.nsysu.edu.tw},
  pdfcontacturl={https://www2.nsysu.edu.tw/iwamoto/},
  pdfcaptionwriter={Sho Iwamoto},
  pdfcopyright={2025 Sho Iwamoto},
  pdflicenseurl={https://www2.nsysu.edu.tw/iwamoto/},
}
\renewcommand\thesubsection{\arabic{subsection}.}
\begin{document}
\maketitle

In university, homework is not just about getting the right answers.
You must think for yourself, explain your ideas, and take responsibility for your work.
Keep the following suggestions in mind, especially when submitting assignments to Sho.
You are also expected to think for yourself and explain your ideas in your own words.

\Remark{Some suggestions may not apply to other instructors, particlaly about the usage of Generative AI, or if the lecture is not in physical science. If unsure, ask your instructor and follow their rules.}


\section*{General Guidelines for Assignments}
\subsection{Submit your own work.}

You are free to use any methods to solve problems, but the final answers \Emph*{need to} be \emph{from you}: based on your thoughts and written in your own words.
\begin{itemize}
  \item Discuss with colleagues, and it's encouraged! But then, you \Emph*{need to} write the names of the people you discussed with: \emph{``I discussed this problem with [Name].''}
  \item \Emph*{Do not} copy answers from others. Looking at someone else's work is not recommended. But if you do, first understand the solution and keep it in your head. Then write the answer out from your own thinking, in your own words.
  \item Using books or websites are encouraged, but then you \Emph*{need to} write down the sources you used. Write the URLs or book titles.
  \item Proper use of AI tools (like ChatGPT) is a very difficult issue in modern university education. See \hrefFN{https://github.com/misho104/LectureCommon/blob/main/Notes/generative_ai.pdf}{Sho's Guidelines for Using Generative AI}.
  If you just copy-and-paste AI output, Sho may consider your work lacks originality, or violates academic ethic, and give you a penalty or a lower grade.
  \item Most importantly, write your final answers based on your own understanding.
\end{itemize}
\OutputNote

\subsection{Show your thinking process.}
Your answer is important, but your process toward the answer is more important than the answer.
Show your steps, formulae, and explanations.
Learn how to explain your thought in English.

\subsection{Try to write like a professional.}
As an advanced suggestion, pay attention to the way of writing.
Professional writing means that others can read, understand, and check your work without confusion and difficulty. In particular,
\begin{miniitemize}
  \item Clarity is the best important. Do not confuse readers.
  \item Use consistent notation throughout your work.
  \item Try to be concise as much as possible, but do not lose clarity.
  \item Clear handwriting is essential.
\end{miniitemize}

\section*{Suggestions for Exercise-Type Assignments}
\subsection{Correct your answers before submission.}
In physics and mathematics courses, exercise-type assignments are given as homework and you will solve the problems to reach the \emph{single correct answer}.
They are sometimes boring but important to cultivate fundamental skills for your further steps.

There, solving problems is not enough---almost useless.
To learn physics, you need to correct the problems you got wrong and learn how to solve them.
Check your work by yourselves before submission.
If you do not check your work before submission, Sho will not give a good evaluation.

\subsection{Do not expect the correct solutions from authority.}
Even you need to check your answer before the submission, you will not have the ``correct solutions.''
Sho never gives you solutions, and textbooks usually do not have complete solutions.

This is because you are an adult, who is expected to be an independent thinker.
Ideally, you need to be confident by yourself, and should not rely on other authorities.
Nobody can tell you the truth\addnote{It is reasonable to assume that anyone offering ``the truth'' is trying to deceive or manipulate you.}, and you need to find it by yourself.

In other words, try \emph{not be taught}, but please \emph{learn by yourselves}.
If you are not sure, it means you do not fully understand the problem and the topic.\addnote{Taken from a Reddit thread: \url{https://www.reddit.com/r/learnmath/comments/7yvo0r/}}

\OutputNote

\subsection{Work together.}
But then, how can you reach the correct solutions?
The best way is to learn by yoursel\Emph{ves}, namely, work together with your colleagues.

Since exercises usually have a single correct answer, working with others just help everyone.
Discussing problems, comparing approaches, and explaining your reasoning to others help you understand the material more deeply.

You can also assume that Sho and the TA as colleagues.
You are encouraged to ask them questions and seek their guidance when needed, particularly during their office hours.


\section*{Suggestions for Open-ended Assignments}
\subsection{Write clearly. Never confuse readers.}
Some assignments, such as critical essay and lab reports, are called \Emph{open-ended assignments}, which do not have a single correct answer but requires your own reasoning, ideas, or researches.
You are often asked to write and submit multiple-page reports.

Clarity is the most important factor in such assignments, because nobody can read your mind.
You need to write clearly, so that others can understand it easily.
In particular, make sure to define each term or symbol in your math, use consistent notation, provide captions and explanations for figures and tables, and explain your finding and thought step by step. ``Long but clear'' writings are far better than ``concise but confusing.''

\subsection{Emphasize originality. Clearly separate your work and others.}
Another important aspect is originality and creativity.
You are asked to develop your own ideas with creativity, but it is not sufficient. You also need to \Emph{emphasize} your originality in your writing.

Recall that nobody can read your mind; nobody can find which part is your original and which part is from literature.
Therefore, you need to clearly indicate your original contributions and ideas in your writing.
For example, you may use explicit phrases such as \emph{``This is my original idea''} to emphasize your originality.
Meanwhile, if you take ideas or discussions from literature, you need to cite it. For citation, please consult the Internet to find appropriate formats and guidelines.



\end{document}
